\documentclass[../main.tex]{subfiles}

\begin{document}

\markboth{}{}
% Plus petite marge du bas pour la quatrième de couverture
% Shorter bottom margin for the back cover
\newgeometry{inner=30mm,outer=20mm,top=40mm,bottom=20mm}

%insertion de l'image de fond du dos (resume)
%background image for resume (back)
\backcoverheader

% Switch font style to back cover style
\selectfontbackcover{ % Font style change is limited to this page using braces, just in case

\selectlanguage{french}

\titleFR{Enrichissement de l’ISA de cœurs de processeurs pour la couche physique des communications mobiles de 5ème génération}

\keywordsFR{4G, 5G, ASIP, SIMD, RISC-V, CCE}

%\abstractFR{Un résumé en français ($\approx 1000 \backsim 1500$ mots).}
\abstractFR{
L'évolution des systèmes IoT défini de nouvelles contraintes sur les standards de communications, à l'image de la 5G. Celle-ci a pour objectif d'assurer la compatibilité entre des millions d'objets connectés hétérogènes et utilise les codes correcteurs d'erreurs (CCE) pour assurer la fiabilité et la qualité des informations. Toutefois, les contraintes applicatives liées à de multiples scénarios possibles exigent de ces systèmes d'être modulaires et adaptables.
Ces travaux proposent la définition et l’étude approfondie d'instructions spécialisées pour étendre l’architecture de processeurs. Cette approche permet d'exploiter la souplesse de programmation des cœurs matériels afin de résoudre ces questions d'adaptabilité et lever ces verrous scientifiques.
Ces jeux d’instructions (ISA) dédiées sont conçues afin de fonctionner sur les processeurs courants (x86, ARM), mais également elles sont prototypées sur cinq cibles de type RISC-V possédant des architectures variées, en 32 et 64 bits. Ces extensions sont étudiées avec des instructions classiques à 2 registres  sources, mais l'étude porte également sur des versions à 3 registres sources. Plusieurs modes de parallélisme de données sont proposés (inter/intra trame), pour des décodeurs scalaires et vectoriels paramétrables.
Les résultats montrent une réduction en latence avec une augmentation en débit pour des décodeurs de CCE utilisés par la 5G avec les codes LDPC et Polaire, les codes LDPC non-binaire (Satellites) et turbo-codes (4G). Les impacts matériels engendrés par l'implantation de ces instructions sont faibles et maîtrisés relativement au cœur ciblé. L’ensemble de ces résultats ont été mesurés sur FPGA et par simulation.
}


\selectlanguage{english}

\titleEN{ISA enrichement of processor cores for the physical layer for the 5th generation mobiles communications}


\keywordsEN{4G, 5G, ASIP, SIMD, RISC-V, ECC, FEC}

%\abstractEN{An abstract in english ($\approx 1000 \backsim 1500$ words).}
\abstractEN{The evolution of IoT systems has defined new constraints on communication standards, such as 5G. The aim of 5G is to ensure compatibility between millions of heterogeneous connected objects and to use error correction codes (ECC) to ensure the reliability and quality of information. However, the application constraints related to multiple possible scenarios require these systems to be modular and adaptable.
This work proposes the definition and in-depth study of specialized instructions to extend the architecture of processors. This approach allows to exploit the flexibility of hardware cores programming in order to solve these adaptability issues and overcome these scientific challenges.
These dedicated ISA (Instruction Set Architecture) are designed to work on common processors (x86, ARM), but they are also prototyped on five RISC-V targets with varying architectures, in 32 and 64 bits. These extensions are studied with classic 2-source register instructions, but the study also covers versions with 3 source registers. Several data parallelism modes are proposed (inter/intra frame), for scalable and vectorizable parametric decoders.
The results show a reduction in latency with an increase in throughput for ECC decoders used by 5G with LDPC and Polar codes, non-binary LDPC codes (Satellites) and turbo-codes (4G). The hardware impacts generated by the implementation of these instructions are low and controlled relatively to the targeted core. All these results have been measured on FPGA and by simulation.
}

}
% Rétablit les marges d'origines
% Restore original margin settings
\restoregeometry

\end{document}
